\documentclass[12pt,a4paper]{article}

\usepackage[a4paper,margin=2.5cm]{geometry}
\usepackage{setspace}
\usepackage[hidelinks]{hyperref}
\usepackage{xurl}
\usepackage[
  backend=biber,
  style=ieee,
  sorting=none,
  doi=true,
  url=true,
  eprint=true
]{biblatex}

\addbibresource{references.bib}

\usepackage[T2A]{fontenc}
\usepackage[utf8]{inputenc}
\usepackage[russian]{babel}

\begin{document}

\begin{titlepage}

{\setstretch{1.0}
\begin{center}
Федеральное государственное автономное образовательное учреждение высшего образования
«Национальный исследовательский университет «Высшая школа экономики»\\
\bigskip
Факультет компьютерных наук\\
Основная образовательная программа\\
Прикладная математика и информатика\\
\end{center}
}

\vspace{8em}

\begin{center}
{\Large КУРСОВАЯ РАБОТА}\\
\textsc{\textbf{
Проект на тему
\linebreak
"Разработка чат-бота для абитуриентов"}}
\end{center}

\vspace{2em}

{\setstretch{1.0}
\hfill\parbox{0.75\textwidth}{
Выполнил студент группы 244, 2 курса,\\
Толмаков Арсений Александрович\\[1em]

Руководитель КР:\\
Андреева Дарья Александровна\\[1em]

}
}

\vspace{\fill}

\begin{center}
Москва 2026
\end{center}

\end{titlepage}
\clearpage
\tableofcontents

\clearpage


\section{Введение}
Разработка чат-ботов для университета становится актуальной задачей в периоды пиковой нагрузки на службу поддержки университета. Большинство вопросов абитуриентов, студентов и остальных пользователей носит повторяющийся характер, и ответами на эти вопросы часто являются правила, требования и прочая информация, которая уже зафиксирована в нормативных документах НИУ ВШЭ. 

Таким образом возникает возможность автоматизировать процесс взаимодействия со студентами и абитуриентами и снизить нагрузку на операторов службы поддержки: удобный чат-бот с понятным для обычного пользователя интерфейсом, и, что самое главное, с проверяемостью ответа, то есть пользователь будет получать ответ со ссылками на конкретный пункт.

Сам бот не будет являться полноценным собеседником -- на него будут наложены ограничения для упрощения взаимодействия с пользователем, такие как: отсутствие диалогов на отвлеченные темы, отсутствие собственного мнения о сотрудниках университета и каких-либо мероприятий/событий, связанных с ним.
\clearpage
\section{Цель и задачи работы}

\subsection*{Цель работы}
Разработать Telegram-бота для снижения нагрузки на службу поддержки НИУ ВШЭ в периоды повышенного спроса (периоды поступления и сессии). Бот будет представлять собой справочный сервис по документу ПОПАТКУС: поиск релевантной информации с указанием конкретного пункта и формирование пояснения по ней. В случае, если информация по запросу не найдена, бот будет давать контактную информацию службы поддержки.

\subsection*{Задачи работы}
\begin{enumerate}
    \item Подготовить заранее типовые запросы и сценарии
    \item Спроектировать структуру работы бота: интерфейс, обработка запроса, модуль семантического поиска по документу и формирование ответа
    \item Разработать механизм извлечения данных из ПОПАТКУС: чтение документа, его разбиение на фрагменты и их привязка к пунктам самого документа
    \item Создать модуль LLM для разъяснения найденной информации и опорой в виде ссылки на источник
    \item Добавить инструкцию для ошибочных запросов -- разъяснение, предложения по исправлению и сообщение об отсутствии информации
    \item Обеспечить фильтрацию ответа бота -- отсутствие сбора личной информации пользователя, отсутствие диалога на отвлеченные темы, отсутствие какого-либо мнения о внутренних сотрудниках, соблюдение политик университета
    \item Интеграция: связать Telegram-интерфейс, систему поиска информации по документу и генерацию ответа LLM в единую систему обработки запросов.
    \item Тестирование работы бота: формирование 50 тестовых вопросов с заранее размеченными эталонными пунктами для рассчета метрик
    \item Добавить возможность оставить отзыв на ответ бота для будущих исправлений
\end{enumerate}
\clearpage
\section{Актуальность}
Основным аргументом в пользу практической актуальности создания чат-бота в Telegram является снижение нагрузки на операторов службы поддержки во время набора новых студентов и периоды сессии. Автоматизация ответов на типовые вопросы значительно снизит время ответа на них и упростит работу службы поддержки.

Если говорить о технической актуальности используемого метода, очень важно учитывать, что ответ на вопрос пользователя должен быть как можно более строгим и при его нахождении его можно будет проверить. Поэтому в данном проекте будет использована архитектура RAG (Retrieval-augmented generation), которая объединяет в себе извлечение релевантных фрагментов из текста документа и формирование ответа языковой моделью на их основе. Такой подход отличается от простой генерации ответа языковой модели тем, что снижает риск возникновения галлюцинаций и додуманной информации. Также повышается надежность работы бота при обновлении документа: нет необходимости в переобучении модели. Это позволит быстрее поддерживать актуальность информации в документе. 

Наконец, такой подход позволит проводить измеримое тестирование ответа и сформулировать критерии качества, что поможет улучшить работу бота.
\clearpage

\section{Обзор литературы}
Рассмотрим существующие методы решения подобных задач.

Основными критериями при выборе метода являются качество понимания текста, написанного пользователем, и достоверность выданной ему информации -- её проверяемость.

Один из классических способов поиска информации в документе -- лексический поиск. То есть система сопоставляет слова из запроса с документом, с которым она работает, и ранжирует результаты. Самый распространненый метод лексического поиска -- BM25~\cite{robertson2009bm25}. Его достоинства в простоте, скорости работы, отсутствии необходимости в обучении и понятном результате. Однако у этого метода есть главное ограничение -- он опирается на совпадение слов, что значительно усложняет задачу в условиях общения с пользователем, который зачастую использует сокращения и разговорную речь.

Существуют способы улучшить лексический поиск: расширение запроса, учёт синонимов и нормализация текста, то есть подгонка под требуемый формат. Однако эти способы ресурсозатратны и требуют поддержки словарей, и при этом все равно не решают проблему полноценно: пользователь может описать свой запрос более свободным языком, чем этого ожидает программа. 

В связи с этими проблемами возник новый способ решения задачи -- методы, умеющие представлять текст семантически. Одним из таких методов стал Word2Vec, позволяющий отображать слова в векторном пространстве, где слова, похожие по смыслу, оказываются рядом ~\cite{mikolov2013word2vec}. Сам способ не решает задачу целиком, однако является важной методологической основой: он значительно расширяет возможности поиска информации по запросу.

Далее рассмотрим ещё один метод семантического поиска - sentence embedding. Довольно популярным подходом к этому методу является SBERT (Sentence Transformers) ~\cite{reimers2019sbert}. С его помощью мы можем получать векторные представления целых предложений и отрывков текста, а затем искать фрагменты, наиболее подходящие запросу пользователя. Также это помогает решить проблему сокращенных слов и разговорных формулировок. Помимо этого, вектора документа можно вычислить заранее, и при запросе пользователя вычислять только вектор запроса и сравнивать с уже заготовленными. 

Но и у метода sentence-embedding существуют ограничения: во многих документах пункты построены однообразно, и при построении векторов может возникнуть ситуация, когда система выберет похожий, однако относящийся к другой теме пункт.

В качестве альтернативы существует еще один метод -- extractive QA~\cite{rajpurkar2016squad}. Система выбирает строгую цитату из документа по запросу. В данном методе снижается вероятность возникновения не относящейся к теме ответа, так как пользователь получит цитату. Но в нашей задаче требуется разъяснение найденной информации, поэтому этот метод используется чаще как подстраховка.

Наконец рассмотрим подробнее используемый нами метод Retrieval-augmented generation ~\cite{lewis2020rag}. Его преимущество в том, что сам документ хранится отдельно от языковой модели, и она поясняет только ту информацию, что выдал релевантный поиск. Это снижает вероятность возникновения неподтвержденных ответов. При обновлении документа достаточно обновить набор фрагментов, пересчитать embeddings, и переиндексировать базу. На практике это сильно упрощает работу архитектуры. 

Еще одним преимуществом, как было упомянуто ранее, является возможность измеримо тестировать качество ответа: насколько часто верный ответ попадает в топы выдачи поиска, и насколько генерируемый ответ соответствует извлеченному фрагменту.

Таким образом, лучшим из обозреваемых решений является архитектура RAG,  как объединение преимуществ семантического поиска и языковой модели.

\clearpage

\section{Результаты первых экспериментов}

В рамках данного проекта единственным источником информации для бота является документ \href{https://www.hse.ru/data/xf/848/118/0135/%D0%9F%D0%BE%D0%BB%D0%BE%D0%B6%D0%B5%D0%BD%D0%B8%D0%B5%20%D0%BE%D0%B1%20%D0%BE%D1%80%D0%B3%D0%B0%D0%BD%D0%B8%D0%B7%D0%B0%D1%86%D0%B8%D0%B8%20%D0%BF%D1%80%D0%BE%D0%BC%D0%B5%D0%B6%D1%83%D1%82%D0%BE%D1%87%D0%BD%D0%BE%D0%B9%20%D0%B0..%D1%81%D0%B8%D1%82%D0%B5%D1%82%D0%B0%20%C2%AB%D0%92%D1%8B%D1%81%D1%88%D0%B0%D1%8F%20%D1%88%D0%BA%D0%BE%D0%BB%D0%B0%20%D1%8D%D0%BA%D0%BE%D0%BD%D0%BE%D0%BC%D0%B8%D0%BA%D0%B8%C2%BB.pdf}{ПОПАТКУС}, что упростит его разработку и тестирование. Будем брать последнюю его версию, которая была принята 26.02.2025 и вступила в силу 01.09.2025.

Перед началом стоит описать базу данных, которая будет использоваться для проведения экспериментов -- ChromaDB ~\cite{chromadb}. Она предоставляет удобную Python API для добавления документов и поиску ближайших соседей в векторном представлении. ChromaDB поддерживает локальный режим работы, что упрощает работу с ней.

В рамках эксперимента ChromaDB была использована для индексации фрагментов ПОПАТКУСа и затем для формирования top-k релевантных фрагментов, чтобы с их помощью генерировать ответ. 

Начинать эксперимент необходимо с дробления документа на страницы, для того, чтобы в будущем ссылаться на них при ответе. Далее проведем очистку и нормализацию текста, удалим артефакты извлечения из формата PDF, и элементы, не несущие смысловой нагрузки.

После этих шагов можно начинать сегментацию - текст каждой страницы разбивается на фрагменты фиксированного размера. Каждому такому фрагменту присваиваются метаданные (уникальный идентификатор, тип информации, номер страницы), которые помогут при поиске и формулировке ответа.
\clearpage

\printbibliography


\end{document}
